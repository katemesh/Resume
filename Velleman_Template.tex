\documentclass[11pt, oneside]{memoir}
\setlrmarginsandblock{0.5in}{3in}{*}
\setulmarginsandblock{0.5in}{0.7in}{*}
\setlength{\sidebarhsep}{0.5in}
\setlength{\sidebarwidth}{2in}
\setsidebarheight{\textheight}
\checkandfixthelayout
\setcounter{secnumdepth}{0}
\nonzeroparskip
\setlength{\parindent}{0pt}
\newcommand{\position}[2]{
     \subsubsection{#1\hfill\mdseries{#2}}}
\usepackage{dtk-logos}

     
     
  \begin{document}
\begin{minipage}[t][0.6in][b]{4in}
    {\fontsize{34pt}{40pt}\selectfont
        \rmfamily\textbf{Kate Mesh, Ph.D.}}
    \vfill
    \textsf{Human communication researcher, mixed-methods analyst for over 10 years}
\end{minipage}
\sidebar{\begin{minipage}[t][0.6in][b]{1in}
    www.katemesh.com
    \vfill
    kate.a.mesh\@gmail.com
\end{minipage}}
\vskip15pt

\sidebar{
    \section{Skills}
    \subsection{Writing}
}

\sidebar{
	\section{Languages}
	English (native)\\
	Spanish (professional)\\
	ASL (professional)\\
	French (reading)\\
	Catalan (elementary)\\
	Eastern Chatino (elementary)\\
}

\sidebar{
    \section{Technologies}

    \subsection{Data collection}
    Eye tracking\\
    Qualtrics
    
    \subsection{Data wrangling \& analysis}
    R\\
    Python\\
    Git
    \subsection{Markup and site generation}
     \LaTeX and \BibTeX\\
      Markdown\\
     Jekyll
}

\section{Experience}
\subsection{Lund University Humanities Lab, Lund University}
\position{Marie Skłodowska-Curie Research Scientist}{2019--present}

Develop and execute research programs, experiment designs, and data analysis to further our understanding of how humans [communicate about their real-world surroundings] [combine speech, gaze, and gestures in face-to-face interaction]. 

Oversee project budget, navigate ethical review board, other institutional tasks based in Sweden

Coordinate an on-site field team to recruit participants and to devise, refine, and implement techniques for qualitative research (culturally-tailored interviews) and quantitative studies (direction-giving experiments fitted to the local culture and physical environment). 

Design experimental tasks for quantitative studies of the \emph{navigation $\cdot$ communication $\cdot$ interaction} strategies identified via discovery research.   

Conduct \emph{foundational $\cdot$ generative $\cdot$ discovery} research via home- and site-visits, using interviews and observations to generate\\ \emph{
$\cdot$ insights about participant recruitment in local populations \\
$\cdot$ spatial maps reflecting how participants navigate the local landscape \\
$\cdot$ insights about when and how participants combine conventional gestures with speech
}

%continuously collecting audiovisual recordings and geoinformation from moving participants during data collection hikes. 

Manage an international team of translators, transcribers, and human action coders to produce rich datasets of multiply-annotated audio-visual materials (5000+ \emph{communicative actions $\cdot$ direction-giving activities} selected and coded to date).  

Coordinate a cross-functional team of language researchers and data scientists to produce quantitative analyses of  \emph{face-to-face interaction $\cdot$ navigation and direction-giving strategies} using multiple statistical approaches (significance testing, regression, generalized linear models).

\subsection{Center for Higher Education in Social Anthropology, Mexico City}
%\subsection{Centro de Investigaciones y Estudios Superiores en Antropología Social}
\position{Hosted Researcher}{2019--present}

\subsection{University of California Institute for Mexico and the United States (UC MEXUS)}
%\subsection{Centro de Investigaciones y Estudios Superiores en Antropología Social}
\position{Co-Instructor, Coexpression and Multimodality in Linguistic Interaction in Mesoamerica, }{2019--present}

\subsection{Sign Language Research Lab, University of Haifa}
%\subsection{Centro de Investigaciones y Estudios Superiores en Antropología Social}
\position{Research Scientist }{2017--2019}

Designed and conducted survey research exploring cross-cultural differences in the interpretation of common hand gestures (target cultures: Israel and Sweden):  


Synthesized current research on hand gestures in each subject culture, then selected video stimuli appropriate to both cultures, producing a stimulus set of NN videos. 

Created open-ended interview questions about the meaning of the gestures, and managed an international team of linguists to translate and localize the questions using MTPE (machine translation post-editing)

Created and executed an interview flow (balanced stimulus presentation) using Qualtrix (\# of participants from each country)

Generated categories of participant responses 

Inter-rater reliability

Results were used to generate hypothesis for additional research studies on the interpretation of conventional gestures



Designed and conducted a quantitative research study comparing how users of two languages (Hebrew and Israeli Sign Language) manage visual attention while playing cooperative games: 

Coordinated with clubs, schools, and elder care centers to NN recruit participants across a broad age span 

Trained language-matched (Hebrew and ISL) game facilitators and oversaw the collection of NN hours of video-recorded game sessions.

Developed a coding scheme to identify strategies for managing visual attention; oversaw a team of translators, translators and coders to generate a study dataset (NN+ attention-managing actions annotated)

Collaborated with a data science team to analyze the results  

\subsection{Linguistics Department, The University of Texas at Austin}
\position{Doctoral Researcher, Course Instructor}{2011--2017}


 

\end{document}   